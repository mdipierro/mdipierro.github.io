\documentclass[12pt]{article}

\def\lines{
\begin{center}
\line(1,0){400} \\
\line(1,0){400} \\
\line(1,0){400} \\
\line(1,0){400} \\
\line(1,0){400} \\
\line(1,0){400} \\
\line(1,0){400} \\
\line(1,0){400} \\
\line(1,0){400} \\
\line(1,0){400} \\
\line(1,0){400} \\
\line(1,0){400} \\
\line(1,0){400} \\
\line(1,0){400} \\
\line(1,0){400} \\
\line(1,0){400} \\
\line(1,0){400} \\
\line(1,0){400} \\
\line(1,0){400} \\
\end{center}
}

\begin{document}

\section*{Final Solution}

\noindent Class: {\bf CSC431} Winter 2013 \\
\noindent Professor: Massimo Di Pierro \\
\noindent Books Allowed: {\bf Yes} \\
\noindent Handwritten Notes Allowed: {\bf No} \\
\noindent Calculator Allowed: {\bf Yes} \\
\noindent Computer Allowed: {\bf No} \\
\noindent Total Problems {\bf 10} \\

Problems are organized by topic. They are not in order of difficulty. Some of them have multiple True or multiple False False answers. For every option circle True or False. Each problem counts for 4 points. An unaswered True/False question will be graded as a wrong answer.

\hskip 5cm

\noindent Student Name: ..................................... \\
\noindent Date: ..................................... \\
\noindent Signature: ..................................... \\

\newpage\section{Problem 1}

\noindent In floating point arithmetic $a+b==a$ when $|b|<\epsilon |a|$. 

\begin{itemize}
\item {\bf True}: In single precision floating point arithmetic $\epsilon \simeq 10^{-7}$
\item {\bf True}: In single precision $a+b==a$ can be true even when $b=1$.
\item {\bf True}: If $a+b==a$ then $n*a + n*b == n*a$ for every $n$
\item {\bf False}: If $a+b==a$ then $a+2*b-b==a$
\end{itemize}

\newpage\section{Problem 2}

\noindent Consider the following functions:

\[
f_1(x) = \frac2{x^2-1} - \frac1{x-1}
\]

\[
f_2(x) = \frac{-1}{1+x}
\]

\noindent Assuming floating point arithmetics, which of the following statments is False:

\begin{itemize}
\item {\bf True}: They are equivalent up to numerical issues
\item {\bf False}: They both diverge when $x=1$
\item {\bf False}: $f_2$ is better than $f_1$ around $x=-1$
\item {\bf False}: $f_1$ is better than $f_2$ around $x=1$
\end{itemize}

\newpage\section{Problem 3}

\noindent Consider these numbers:
$x_0 = 1+a$, $x_1 = 1-a$ for $x_i=1$ for $i = 2...N-1$

\noindent and $a=10^{-3}$. The average is $\mu = 1$ and the variance is $\sigma^2 = 2a^2/N$.

\noindent Assume single precision floating point arithmetics and assume you compute the variance using the formula:
\[
\sigma^2 = (\frac{1}{N} \sum_i x^2) - \mu^2
\]
\noindent For which value of $N$ will you get an incorrect value for the variance? Explain your answer.

\begin{eqnarray}
\sigma^2 &<& \epsilon \mu^2  \\
2a^2/N &<& \epsilon \mu^2 \\
N &>& 2a^2/(\epsilon \mu^2) \\
N &>& 20
\end{eqnarray}

\newpage\section{Problem 4}

\noindent Consider the following matrix:

\[
\left(
\begin{tabular}{ccc}
1 & 0 & 3 \\
1 & 3 & 0 \\ 
0 & 1 & 3 
\end{tabular}
\right)
\]

\noindent Compute the inverse in 6 steps. Show your steps.

\[
\left(
\begin{tabular}{ccc}
3/4 & 1/4 & -3/4 \\
-1/4 & 1/4 & 1/4 \\
1/12 & -1/12 & 1/4
\end{tabular}
\right)
\]

\newpage\section{Problem 5}

\noindent Consider the following $2 \times 2$ matrix:

\[
\left(
\begin{tabular}{cc}
1 & a \\
a & 2 
\end{tabular}
\right)
\]

\noindent For which values of $a$ is it positive definite? Show your steps.

It is symmetric. It is positive definite if the Chlwesky exists. If we call the matrix $M$:

\begin{eqnarray}
l_{00} &=& \sqrt{m_{00}} = 1\\
l_{01} &=& m_{01}/l_{01} = a\\
l_{11} &=& \sqrt{m_{11}-l_{01}^2} = \sqrt{2-a^2}
\end{eqnarray}
Requires $|a|<\sqrt{2}$.


\newpage\section{Problem 6}

\noindent Consider the following algorithms:

\begin{verbatim}
def D(f,h=1e-4):
    return lambda x: (f(x+h)-f(x-h))/(2.0*h)

def P(f,x,ns=10):
    for k in range(ns): x = x - f(x)/D(f)(x)
    return x

def Q(f,x,ns=10):
    for k in range(ns): x = x - f(x)/D(D(f))(x)
    return x
\end{verbatim}

\begin{itemize}
\item {\bf True}: $P(f)$ finds $x$ which solves $f(x)=0$ using Newton method.
\item {\bf False}: $Q(f)$ finds $x$ which solves $f'(x)=0$ using Newton method.
\item {\bf False}: $P(f)$ always converges and converges to the correct result.
\item {\bf False}: Each call to $P$ calls the function $f$, $2\times\mathrm{ns}$ times.
\end{itemize}

\newpage\section{Problem 7}

\noindent What is the output of

\begin{verbatim}
def f(x): return x**3-x-1
print P(f,x=0,ns=2)
\end{verbatim}
Show your steps.

\begin{verbatim}
x1 = x0 - f(x0)/f'(x0) = -1
x2 = x1 - f(x1)/f'(x1) = -0.5
\end{verbatim}

\newpage\section{Problem 8}

\noindent Consider the following algorithms:

\begin{verbatim}
def R(f,x,ap=1e-5,rp=1e-5,h=1e-4):
    while True:
        (x_old, x) = (x, x - f(x)*h/(f(x)-f(x-h)))
        if abs(x-x_old)<max(ap,rp*abs(x)): return x

def T(f,x,ap=1e-5,rp=1e-5,h=1e-4):
    fx = f(x)
    (x_old, fx_old, x) = (x, fx, x - fx*h/(fx-f(x-h)))
    while True:
        fx = f(x)
        (x_old, fx_old, x) = (x, fx, x - fx*(x-x_old)/(fx-fx_old))
        if abs(x-x_old)<max(ap,rp*abs(x)): return x
    return x
\end{verbatim}

\begin{itemize}
\item {\bf True}: The two algorithms solve the same problems in different ways
\item {\bf False}: The two algorithms always return a solution.
\item {\bf True}: The two algorithms can go into an infinite loop
\item {\bf False}: {\tt R} is more efficient than {\tt T} (less calls to function $f$ for same precision).
\end{itemize} 

\newpage\section{Problem 9}

\noindent Consider the following three points:

\begin{tabular}{cc} \hline
$t_i$ & $y_i$ \\ \hline
0 & 2 \\
1 & 5 \\
2 & 6
\end{tabular}

\noindent You perform a linear fit using the linear least square algorithm with $y(t) = c_0 + c_1 t$. The output is given by $c = (A^t A)^{-1} A^t y$. Where $A_{i0} = 1$, $A_{i1}=t_i$.

\noindent Using the fact that 

\[
\left( \begin{tabular}{cc} a & 1\\ 1 & 1\end{tabular}\right)^{-1} = \frac1{a-1}
\left( \begin{tabular}{cc} 1 & -1\\ -1 & a\end{tabular}\right)
\]

\noindent determine the value of the coefficients $c$. Show your steps.

\begin{eqnarray}
A &=& 
\left(
\begin{tabular}{cc}
1 & 0 \\
1 & 1 \\
1 & 2
\end{tabular}
\right) \\
y &=& 
\left(
\begin{tabular}{c}
2 \\
5 \\
6
\end{tabular}
\right) \\
A^t A &=& 3
\left(
\begin{tabular}{cc}
1 & 1 \\
1 & 5/3
\end{tabular}
\right) \\
(A^t A)^{-1} &=& \frac13 \frac3/2
\left(
\begin{tabular}{cc}
5/3 & -1 \\
-1 & 1
\end{tabular}
\right) \\
A^t y &=&
\left(
\begin{tabular}{c}
13 \\
17 
\end{tabular}
\right) \\
c &=& (A^t A)^{-1} A^t y =
\left(
\begin{tabular}{c}
7/3 \\
2
\end{tabular}
\right)
\end{eqnarray}

\newpage\section{Problem 10}

\begin{itemize}
\item {\bf True}: The Newton solver is likely to fail when $f'(x) \simeq 0$ in proximity of of the solution.
\item {\bf False}: The Newton optimizer is likely to fail when $f''(x) >> 0$ in proximity of the solution. 
\item {\bf True}: $f'(x)=0$ and $f''(x)>0$ is a relative minimum of the function $f$.
\item {\bf False}: $f'(x)=0$ is either relative minimum or a maximum of the function $f$.
\end{itemize}

\end{document}
