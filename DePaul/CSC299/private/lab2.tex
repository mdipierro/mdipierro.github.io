\documentclass[12pt]{article}

\usepackage{graphicx}
\usepackage{url}
\begin{document}

\section{CSC299 - Fall 2015 - Lab Assignment 2}

\subsection*{IMPORTANT INSTRUCTIONS}

Use this URL to verify your progress:
\begin{verbatim}
https://mdp.cdm.depaul.edu/csc299
\end{verbatim}

Login into:
\begin{verbatim}
mdp.cdm.depaul.deu
\end{verbatim}

Under your csc299 folder create a new folder called {\tt lab02} and, under the latter, create a new file called {\tt README.md}. This file should start with:

\begin{verbatim}
Name: <yourname>
StudentId: <yourstudentid>
Email: <youremail>
\end{verbatim}

Put the answers to the questions belong in the file
{\tt ~/csc299/lab02/README.md}.

After completing each task below remember to do:

\begin{verbatim}
cd ~/csc299
git add .
git commit -a -m "task completed"
git push
\end{verbatim}

else I will not receive your work.

\subsection{Task 1 (3 points)}

Change folder and go under {\tt ~/csc299/lab02}.

Create a program called {\tt program1.py} which contains:

\begin{verbatim}
from random import randint
myfile = open('data.csv','w')
for k in range(1000):
    myfile.write(', '.join([str(randint(0,1000)) for i in '0123'])+'\n')
\end{verbatim}

Run it and make sure it creates a CSV file which contains 4 columns and 1000 rows of random numbers.

\subsection{Task 2 (4 points)}

Write a program in the same folder as the one above, called {\tt program2.py} which when called with:

\begin{verbatim}
python program2.py
\end{verbatim}

reads the above {\tt data.csv} file and computes the totals of each column and stores in a file {\tt sums.csv} the 4 numbers in one line separated by a comma. For example:

\begin{verbatim}
12344, 21123, 41212, 31244
\end{verbatim}

\subsection{Task 3 (4 points)}

Write a program in the same folder as the one above, called {\tt program3.py} which takes no input. The program will produce a file called {\tt bincount.csv} which contains:

\begin{verbatim}
1, 10
2, 7
3, 10
...
\end{verbatim}

The first column is a progressive number and the second column counts the number of files in {\tt /bin/} which have a filename length equal to the value in the first column. Example 10 files of length 1, 7 files of length 2, etc.

Tip: use {\tt os.listdir(...)}.
\end{document}
