\documentclass[12pt]{article}

\usepackage{graphicx}
\usepackage{url}
\usepackage{upquote}
\begin{document}

\section{CSC299 - Fall 2015 - Lab Assignment 5}

\subsection*{IMPORTANT INSTRUCTIONS}

\noindent Use this URL to verify your progress:
\begin{verbatim}
https://mdp.cdm.depaul.edu/csc299
\end{verbatim}

\noindent Login into:
\begin{verbatim}
mdp.cdm.depaul.edu
\end{verbatim}

\noindent Under your csc299 folder create a new folder called {\tt lab05} and, under the latter, create a new file called {\tt README.json}. This file should contain:

\begin{verbatim}
{"student_id": "<yourstudentid>", "name": "<yourname>", "email": "<youremail>"}
\end{verbatim}

\noindent After completing each task below remember to do:

\begin{verbatim}
git add README.json
git add .
git commit -a -m "task completed"
git push
\end{verbatim}

\noindent  else I will not receive your work.

\noindent ATTENTION: Files are different for each student.

\subsection{Task 1 (3 points)}

\noindent  Create a new folder {\tt ~/csc299/lab05} and {\tt cd} under that folder.

\noindent Download the document at this URL

{\tiny
\begin{verbatim}
http://odata.cdm.depaul.edu/Cdm.svc/Courses?$orderby=CatalogNbr&$filter=EffStatus%20eq%20%27A%27%20and%20SubjectId%20eq%27CSC%27
\end{verbatim}}

\noindent into a file called {\tt CSC.xml}. The file has the following structure

\begin{verbatim}
...
      <m:properties>
        <d:CrseId>001362</d:CrseId>
        <d:SubjectId>CSC</d:SubjectId>
        <d:CatalogNbr>200</d:CatalogNbr>
        ...
      </m:properties>
      <m:properties>
        ...
      </m:properties>
...
\end{verbatim}

\noindent Use BeautifulSoup to extract the content of the first {\tt <m:properties>} block and dump it to a file {\tt CSC.properties.1.xml}. Call your program {\tt program51.py}.

\subsection{Task 2 (3 points)}

\noindent Use BeautifulSoup to extract the content of the first {\tt <m:properties>} block and convert it to a python dictionary as follows:

\begin{verbatim}
      <m:properties>                     -> {
        <d:CrseId>001362</d:CrseId>      ->   'CrseId': '001362',
        <d:SubjectId>CSC</d:SubjectId>   ->   'SubjectId': 'CSC',
        <d:CatalogNbr>200</d:CatalogNbr> ->   'CatalogNbr': '200',
        ...                                   ...
      </m:properties>                    -> }
\end{verbatim}

\noindent and use simplejson to store the dictionary into a file {\tt CSC.properties.1.json}. Call your program {\tt program52.py}.

\noindent {\bf Tip:} Consider this code:
\begin{verbatim}
soup = BeautifulSoup(page)
soup.findAll(name=re.compile('^d\:\w+'))
\end{verbatim}

\noindent The findall finds all tags with a name starting with {\tt d:}{\it something}.

\noindent {\bf Tip}: Consider this code:

\begin{verbatim}
import simplejson
obj = {'hello':'world'}
simplejson.dump(obj, open('obj.json','w'))
copy = simplejson.load(open('obj.json','r'))
print obj == copy
\end{verbatim}

\noindent The simplejson library allows to dump an object to a file into JSON format and load it back.

\subsection{Task 3 (3 points)}

\noindent Use BeautifulSoup to extract the content of all properties items into a list of dictionaries and store the list into a JSON file {\tt CSC.json}. Call your program {\tt program53.py}.

\subsection{Task 4 (2 points)}

\noindent Write a program called {\tt program54.py} that takes as command line argument some keywords for example:

\begin{verbatim}
python program54.py Python > python.log
\end{verbatim}

Run the above command!

\noindent and outputs a list of course names and course descriptions that include the keywords (in the example ``Python'') in the course name. 

\noindent {\bf Tip:} Consider this code:

\begin{verbatim} 
import sys
text = 'this is a test'
keywords = sys.argv[1:]
if keywords and all(k in text for k in keywords):
    print text
\end{verbatim}

\noindent It reads the command line arguments into keywords and prints text if text contains all the keywords.

\end{document}
