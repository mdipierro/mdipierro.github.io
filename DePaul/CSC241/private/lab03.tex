\documentclass[12pt]{article}
\usepackage{upquote}

\begin{document}

{\Large CSC241 Winter 2014 - Lab Assignment 3}

\section{Program 1 (2 points)}

\noindent
The DNA of living creatures consists of a double helix of four types of nucleotides,               
commonly referred to ATGC. They came in pairs A-T, T-A, G-C, C-G.                                  
This means that a ATGC sequence of one strand corresponds to the TACG on the other strand.         
Write a program that given one strand of DNA, in the form of one input string,                     
computes the complementary strand. For example:
\begin{verbatim}
input: ATGCATGC
output: TACGTACG
\end{verbatim}
The program file must be called {\tt a3p1.py}

\noindent When done you can grade it automatically with
\begin{verbatim}
$ grade a3p1.py
\end{verbatim}

\section{Program 2 (2 points)}

\noindent 
Consider a chessboard. Each cell in the chessboard has two coordinates (x,y) for example (0,0),    
(0,1),(1,1),(2,3),etc. Now consider a simple game where you start on cell (0,0). The game          
repeatedly asks you to input a string. If you input U it moves up (x=x+1),                         
if you input D it moves down, if you input L it moves left (y=y-1),                                
if you move input R it moves right, and if you input Q it prints the coordinates                   
of the cell where where you are, in the form ``(4,5)'' (notice brackets but not spaces).             
\begin{verbatim}
input: U
input: U
input: U
input: L
input: Q
output: (3,-1)
\end{verbatim}
    
The program must be called ({\tt a3p2.py}) 

\noindent When done you can grade it automatically with
\begin{verbatim}
$ grade a3p2.py
\end{verbatim}

\section{Program 3 (2 points)}

\noindent Write a program ({\tt a3p3.py})
that inputs a string and prints 'True' if the world is a palindrome,               
'False' otherwise.
\begin{verbatim}
$ grade a3p3.py
\end{verbatim}

\section{Program 4 (2 points)}
\noindent Write a program ({\tt a3p4.py})
Which contains the following code:
\begin{verbatim}
def getMetamorphosis():
    return open('/tmp/kafka.txt').read()

def countWord(s, p):
    counter = 0
    for k in range(0,len(s)-len(p)):
        # FILL HERE
    return counter

print(countWord(getMetamorphosis(), 'Gregor'))
\end{verbatim}
This program is supposed to count the number of instances of the word ``Gregor'' in Kafka's Metamorphosis. Complete the body of the function countWord.

\begin{verbatim}
$ grade a3p4.py  
\end{verbatim}
\end{document}
