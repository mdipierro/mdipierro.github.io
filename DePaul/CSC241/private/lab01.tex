\documentclass[12pt]{article}
\usepackage{upquote}

\begin{document}

{\Large CSC241 Winter 2014 - Lab Assignment 1}

\section{Examples}

\noindent Consider the following Python program:
\begin{verbatim}
s = input('write something: ')
print(len(s))
\end{verbatim}
\noindent It prints the length of the string you input.

\noindent Consider the following Python program:
\begin{verbatim}
s = input('write something: ')
n = len(s)
print(s[n-1])
\end{verbatim}
\noindent It prints the last character of the string you input.

\noindent Consider the following Python program:
\begin{verbatim}
n = int(input('write a number: '))
for k in range(n):
    print k
\end{verbatim}
\noindent It prints all numbers between 0 and n.

\noindent Consider the following Python program:
\begin{verbatim}
n = int(input('write a number: '))
for k in range(n):
    print 2*k
\end{verbatim}
It prints all the even numbers between 0 and $2n$ (not including $2n$).

\noindent Consider the following Python program:
\begin{verbatim}
s = input('write something: ')
w = ''
for k in range(len(s)):
    if s[k] != 'a':
        w = w + s[k]
print(w)
\end{verbatim}
It asks you to input something and prints is back to you after remoging the character 'a' from the text you typed.

\section{Program 1 (2 points)}

\noindent Write a program ({\tt a1p1.py}) 
that asks you to input a number n (an integer)
and prints the squares of all numbers between 0 and n.
The program file must be called {\tt a1p1.py}

\noindent When done you can grade it automatically with
\begin{verbatim}
$ grade a1p1.py
\end{verbatim}

\section{Program 2 (2 points)}

\noindent Write a program ({\tt a1p2.py}) 
that asks you to input a string and prints the string reversed.

\noindent When done you can grade it automatically with
\begin{verbatim}
$ grade a1p2.py
\end{verbatim}

\section{Program 3 (2 points)}

\noindent Write a program ({\tt a1p3.py})
that lets you input a string and prints the string after
replacing all letters 'a' with letter 'o' and all letters 'o' with letter 'a'.
When done you can grade it automatically with
\begin{verbatim}
$ grade a1p3.py
\end{verbatim}

\section{Program 4 (2 points)}
\noindent Write a program ({\tt a1p4.py})
that inputs a string and returns the string converted 
to morse code (leaving spaces invariant).
For example input 'aa bb' should produce output '.-.- -...-...'
When done you can grade it automatically with
\begin{verbatim}
$ grade a1p4.py  
\end{verbatim}

\section{Program 5 (3 points extra credit)}
\noindent Write a program ({\tt a1p5.py})
that inputs a string in Morse code and converts it back to normal text.
For example input '.-.- -...-...' should produce output 'aa bb'

\end{document}
