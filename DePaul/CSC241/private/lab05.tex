\documentclass[12pt]{article}
\usepackage{upquote}

\begin{document}

{\Large CSC241 Winter 2014 - Lab Assignment 5}

\section{Program 1 (2 points)}

\noindent
Write a program {\tt a5p1.py} that perform the following three steps: 1) repeatedly asks you to input a number (an integer number) until you enter an empty string, and stores all the numbers in a list A. 2) repeatedly asks you to input another set of numbers until you enter an empty string, and stores all the numbers in a list B. 3) prints True if the maximum number in A is the same as the maximum number in B and if the minimum number in A is the same as the minimum number in B, else it prints False. For example:
\begin{verbatim}
> 2
> 3
> 8
> 
> 8
> 5
> 2
>
True
\end{verbatim}
\noindent When done you can grade it automatically with
\begin{verbatim}
$ grade a4p1.py
\end{verbatim}


\section{Program 2 (2 points)}

\noindent
Write a program {\tt a4p2.py} that asks you for a filename, opens and reads the file, splits the content into words (use the str.split() function), and prints all the words that have a length equal to exactely 4 chacaters. Words should be printed in the order they are found, one per line. Some words may be repeated.

\noindent When done you can grade it automatically with
\begin{verbatim}
$ grade a4p2.py
\end{verbatim}


\section{Program 3 (2 points)}

\noindent
Write a program {\tt a4p3.py} that repeatedly asks you to input a number (float) until 
you enter an empty string. It then computes the average of those numbers and prints back, one per line, those number that dist from the average no more than 10. Example:

\begin{verbatim}
> 1.2
> 2.3
> 30.9
> 5.1
> 3.1
>
1.2
2.3
5.1
3.1
\end{verbatim}

\noindent When done you can grade it automatically with
\begin{verbatim}
$ grade a4p3.py
\end{verbatim}

\section{Program 4 (2 points)}

\noindent
Write a program {\tt a5p4.py} that asks you to input a number, then asks you to input the units of that number (for example: m, cm, km), then asks you to input another type of units (for example: m, cm, km), then prints the number converted to the latter units. Example:
\begin{verbatim}
> 34.5
> km
> cm
3450000.0
\end{verbatim}
\noindent Important. Use the following syntax to print a floating point {\tt n} so that it is properly formatted:
\begin{verbatim}
print("%f" % n)
\end{verbatim}
\noindent You can restrict to conversions between cm,m,km only, but you are free try other units as long as cm,m,km are supported.

\noindent When done you can grade it automatically with
\begin{verbatim}
$ grade a4p4.py
\end{verbatim}

\section{Program 5 (2 points extra credit)}

Consider the following code:
\begin{verbatim}
from random_password import check_password
\end{verbatim}
It makes up a random password consisting of 6 characters in the set abcde (for example baaecd) but does not tell you what the password is.

Now consider this code:
\begin{verbatim}
guess = 'acdbde'
n = check_password(guess)
\end{verbatim}
It takes your guess and returns True if you guessed the correct password, else it tells you how many consecutive characters from your guess match the correct password, starting from the first caracter. Example:
\begin{verbatim}
(computer thinks of cdebaa)
>>> from random_password import check_password
>>> print(check_password('cdeaba')
3
\end{verbatim}

Write a program {\tt a5p5.py} that finds and prints the correct password. Notice the password is different every time you run your program.
\end{document}
