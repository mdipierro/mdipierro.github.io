\documentclass[12pt]{article}
\usepackage{upquote}

\begin{document}

{\Large CSC241 Winter 2014 - Lab Assignment 7}

\section{Program 1 (2 points)}

\noindent
The module {\tt io241} exposes the following functions:
\begin{verbatim}
io241.get_page(url)
\end{verbatim}
which given a URL returns the content of the web page, and
\begin{verbatim}
io241.parse_emails(page)
\end{verbatim}
which given a web page returns a list of the email addresses found in the page.

Write a program {\tt a7p1.py} lists all the email addresses found at the web page
\begin{verbatim}
http://www.cdm.depaul.edu/about/Pages/People/Administration.aspx
\end{verbatim}
Grade it as usual:
\begin{verbatim}
$ grade a7p1.py
\end{verbatim}


\section{Program 2 (2 points)}

The module {\tt io241} exposes the following functions:
\begin{verbatim}
io241.world_news()
\end{verbatim}
and returns a list of news titles from NPR.rg.
\noindent
Write a program {\tt a7p2.py} that lists all the titles,
asks you to enter a number and prints the corresponding title in upper case. Example:
\begin{verbatim}
0) First man in the moon
1) Water on Mars
2) Flood on Jupiter
: 1
WATER ON MARS
\end{verbatim}

\noindent When done you can grade it automatically with
\begin{verbatim}
$ grade a7p2.py
\end{verbatim}

\section{Program 3 (2 points)}

\noindent The International Civil Aviation Organization (ICAO) alphabet assigns code words to the letters of the English alphabet acrophonically (Alfa for A, Bravo for B, etc.) so that critical combinations of letters (and numbers) can be pronounced and understood by those who transmit and receive voice messages by radio or telephone regardless of their native language, especially when the safety of navigation or persons is essential. Here is a Python dictionary covering one version of the ICAO alphabet:
\begin{verbatim}
d = {'a':'alfa', 'b':'bravo', 'c':'charlie', 'd':'delta', 'e':'echo', 
     'f':'foxtrot', 'g':'golf', 'h':'hotel', 'i':'india', 'j':'juliett', 
     'k':'kilo', 'l':'lima', 'm':'mike', 'n':'november', 'o':'oscar', 
     'p':'papa', 'q':'quebec', 'r':'romeo', 's':'sierra', 't':'tango', 
     'u':'uniform', 'v':'victor', 'w':'whiskey', 'x':'x-ray', 
     'y':'yankee', 'z':'zulu'}
\end{verbatim}
Write a program that inputs a string and converts them to ICAO as follows:
\begin{verbatim}
input: ab cd
output: alpha bravo   charlie delta
\end{verbatim}

\noindent When done you can grade it automatically with
\begin{verbatim}
$ grade a7p3.py
\end{verbatim}

\section{Program 4 (2 points)}

\noindent
The third person singular verb form in English is distinguished by the suffix -s, which is added to the stem of the infinitive form: run -> runs. A simple set of rules can be given as follows:
\begin{itemize}
\item If the verb ends in y, remove it and add ies
\item If the verb ends in o, ch, s, sh, x or z, add es
\item By default just add s
\end{itemize}

Write a program {\tt a7p4.py} that takes as input an english verb in infinite form and returns its plural form. Example:
\begin{verbatim}
input: love
output: loves
\end{verbatim}
\noindent When done you can grade it automatically with
\begin{verbatim}
$ grade a7p4.py
\end{verbatim}

\end{document}
