\documentclass[12pt]{article}
\usepackage{upquote}

\begin{document}

{\Large CSC241 Winter 2014 - Lab Assignment 6}

\section{Program 1 (2 points)}

\noindent
Write a program {\tt a6p1.py} which contains the following:
\begin{verbatim}
a = input('write some text:').split()

maximum = 0
for element in range(a):
    if len(element)>maximum:
       maximum = element

print 'The longest word in the text is %s characters long' % maximum
\end{verbatim}
This program has some bugs. Try understand what it is supposed to do, and fix it so that it produces the correct result.
\noindent When done you can grade it automatically with
\begin{verbatim}
$ grade a6p1.py
\end{verbatim}


\section{Program 2 (2 points)}

\noindent
Complete the following program {\tt a6p2.py}:
\begin{verbatim}
def split_into_words(text):
    # fill here
    return words

def find_length_of_longest_word(words):
    # fill here
    return length

def main():
    text = input('write some text:')
    length = find_length_of_longest_word(split_into_words(text))
    print(text)

main()
\end{verbatim}
\noindent When done you can grade it automatically with
\begin{verbatim}
$ grade a6p2.py
\end{verbatim}


\section{Program 3 (2 points)}

\noindent
Write a program {\tt a6p3.py} that works like program 2 but instead of printing the length of the longest word, it prints the first word found of that maximum length. 
\begin{verbatim}
\end{verbatim}
\noindent When done you can grade it automatically with
\begin{verbatim}
$ grade a6p3.py
\end{verbatim}

\section{Program 4 (2 points)}

\noindent
Write a program {\tt a6p4.py} that takes some text as input and replaces each word with its reversed. For example
\begin{verbatim}
input: this is a test of some input
output: siht si a tset fo emos tupni
\end{verbatim}
\noindent When done you can grade it automatically with
\begin{verbatim}
$ grade a6p4.py
\end{verbatim}

\section{Program 5 (2 points extra credit)}
Write a program that inputs some text and encrypts it as follows:
\begin{itemize}
\item the order of charaters in each word is reveresed (as in program 4)
\item the order of the words is reversed
\item each alphanumeric character is converted with the next characters in the alphabet using this function:
\begin{verbatim}
def next_char(c):
    A,a,k,z,Z = ord('A'),ord('a'),ord(c),ord('z'),ord('Z')
    if A<=k<=Z: return chr(A+(k-A+1)%26)
    elif a<=k<=z: return chr(a+(k-a+1)%26)
    else: return z
\end{verbatim}
\end{itemize}
When done, write a program that decrypts encrypted messages.

\end{document}
