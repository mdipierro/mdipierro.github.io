\documentclass[12pt]{article}
\usepackage{upquote}

\begin{document}

{\Large CSC241 Winter 2014 - Lab Assignment 8}

\section{Program 1 {\tt a8p1.py} (1 points)}

Write a version of a palindrome recogniser that accepts a file name from the user, reads each line, and prints the line to the screen if it is a palindrome. 

\section{Program 2 {\tt a8p2.py} (1 points)}

According to Wikipedia, a semordnilap is a word or phrase that spells a different word or phrase backwards. (``Semordnilap'' is itself ``palindromes'' spelled backwards.) Write a semordnilap recogniser that accepts a file name (pointing to a list of words) from the user and finds and prints all pairs of words that are semordnilaps to the screen. For example, if ``stressed'' and ``desserts'' is part of the word list, the the output should include the pair ``stressed desserts''. Note, by the way, that each pair by itself forms a palindrome!

\section{Program 3 {\tt a8p3.py} (1 points)}

Write a procedure {\tt char\_freq\_table()} that, when run in a terminal, accepts a file name from the user, builds a frequency listing of the characters contained in the file, and prints a sorted and nicely formatted character frequency table to the screen.

\section{Program 4 {\tt a8p4.py} (1 points)}

Write a program that takes two strings as input. The first is a filename. It then open the file, reads its lines and prints the lines that contain the second input string.

\section{Program 5 {\tt a8p5.py} (2 points)}

Write a program that takes two strings as input. The first is a filename. It then open the file, reads its lines and prints the lines that contain any of the characters in the second string.

\section{Program 6 {\tt a8p6.py} (2 points)}

Consider the following Python dictionary (also found in file {\tt /tmp/ami.py}):
\begin{verbatim}
AMI = {'Ala/A':['GCT','GCC','GCA', 'GCG'],
       'Leu/L':['TTA', 'TTG', 'CTT', 'CTC', 'CTA', 'CTG'],
       'Arg/R':['CGT', 'CGC', 'CGA', 'CGG', 'AGA', 'AGG'],
       'Lys/K':['AAA', 'AAG'],
       'Asn/N':['AAT', 'AAC'],
       'Met/M':['ATG'],
       'Asp/D':['GAT', 'GAC'], 
       'Phe/F':['TTT', 'TTC'],
       'Cys/C':['TGT', 'TGC'],
       'Pro/P':['CCT', 'CCC', 'CCA', 'CCG'],
       'Gln/Q':['CAA', 'CAG'],
       'Ser/S':['TCT', 'TCC', 'TCA', 'TCG', 'AGT', 'AGC'],
       'Glu/E':['GAA', 'GAG'], 
       'Thr/T':['ACT', 'ACC', 'ACA', 'ACG'],
       'Gly/G':['GGT', 'GGC', 'GGA', 'GGG'], 
       'Trp/W':['TGG'],
       'His/H':['CAT', 'CAC'], 
       'Tyr/Y':['TAT', 'TAC'],
       'Ile/I':['ATT', 'ATC', 'ATA'],
       'Val/V':['GTT', 'GTC', 'GTA', 'GTG']}
\end{verbatim}
It maps the name of aminoacids to the corresponding codons (DNA subsequences of size 3).
Write a program that reads as input a DNA string, breaks it into codons (substrings of size 3) and counts the corresponding aminoacids. It should output aminoacids (sorted alphabetically) and their frequency (separated by a colon, one per line). Example:
\begin{verbatim}
input> ATGATGAAA
output>
Lys/K: 1
Met/M: 2
\end{verbatim}
You can use the following syntax so loop overthe keys of a dicationary {\tt d} in sorted order:
\begin{verbatim}
for key in d: print(k,d[key])
\end{verbatim}
\end{document}
