\documentclass[12pt]{article}
\usepackage{upquote}

\begin{document}

{\Large CSC241 Winter 2014 - Lab Assignment 4}

\section{Program 1 (2 points)}

\noindent
Complete the following program {\tt a4p1.py}
\begin{verbatim}
def histogram(a):
    # complete this...

a = []
while True:
    s = input('number:')
    if not s: break
    a.append(int(s))

histogram(a)     
\end{verbatim}
Complete the function {\tt histogram()} 
that takes a list of integer numbers and                      
prints a histogram to the screen. For example, histogram([4, 9, 7])                          
should print the following:                                                                  
\begin{verbatim}
****                                                                                         
*********                                                                                    
*******
\end{verbatim}

Tip: {\tt s='*'*5} is the same as {\tt s='*****'}.
\noindent When done you can grade it automatically with
\begin{verbatim}
$ grade a4p1.py
\end{verbatim}


\section{Program 2 (2 points)}

\noindent
Write a program {\tt a4p2.py} 
that repeatedly asks to type a number until you enter an empty string.                   
It then prints a string containing the same numbers sorted, separated by a comma,             
without spaces. 
For example:

\begin{verbatim}
input: 
3
8
2
quit

output:
2,3,8
\end{verbatim}

Tip: store the numbers in a list use the sort method of a list.
The following lines can be useful:
\begin{verbatim}
a = [3,2,1]
a.sort()
s = ''+str(a[0])+','+str(a[1])+','+str(a[2])
\end{verbatim}

\noindent When done you can grade it automatically with
\begin{verbatim}
$ grade a4p2.py
\end{verbatim}


\section{Program 3 (2 points)}

\noindent
Write a program {\tt a4p3.py} 
that asks you for a file name, open the file and prints the average          
length of the words in the file.

Tip: The following line could be useful:
\begin{verbatim}
words = open(filename).read().split()
\end{verbatim}

\noindent When done you can grade it automatically with
\begin{verbatim}
$ grade a4p3.py
\end{verbatim}

\section{Program 4 (2 points)}

\noindent
Complete the following program {\tt a4p4.py}.
\begin{verbatim}
def overlapping(a,b):
    # complete this

a = input('string 1:').split()
b = input('string 2:').split()
print(overlapping(a,b))
\end{verbatim}
Complete the function {\tt overlapping()\}}. 
It takes two lists and returns True if they have at least one member in common, False otherwise. 

\noindent When done you can grade it automatically with
\begin{verbatim}
$ grade a4p4.py
\end{verbatim}

Extra credit. The programs at point 3 and 4 use split. That means that they will break ``this. test'' into [``this.'', ``test'']. See the ``.''? It should not be there. Try write your own split function that identifies the words and properly ignores symbols like ``.''.

Extra Extra credit. Modify the histogram program so that the bars are normalized and the longest one contains always 40 stars.

Extra Extra Extra credit. Modify the histogram program and plot the histograms vertically instead of horizontally.
\end{document}
